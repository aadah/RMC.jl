%%%%%%%% ICML 2023 EXAMPLE LATEX SUBMISSION FILE %%%%%%%%%%%%%%%%%

\documentclass{article}

% Recommended, but optional, packages for figures and better typesetting:
\usepackage{microtype}
\usepackage{graphicx}
\usepackage{subfigure}
\usepackage{booktabs} % for professional tables

% hyperref makes hyperlinks in the resulting PDF.
% If your build breaks (sometimes temporarily if a hyperlink spans a page)
% please comment out the following usepackage line and replace
% \usepackage{icml2023} with \usepackage[nohyperref]{icml2023} above.
\usepackage{hyperref}


% Attempt to make hyperref and algorithmic work together better:
\newcommand{\theHalgorithm}{\arabic{algorithm}}

% Use the following line for the initial blind version submitted for review:
\usepackage{icml2023}

% If accepted, instead use the following line for the camera-ready submission:
% \usepackage[accepted]{icml2023}

% For theorems and such
\usepackage{amsmath}
\usepackage{amssymb}
\usepackage{mathtools}
\usepackage{amsthm}

\usepackage{multicol}

% if you use cleveref..
\usepackage[capitalize,noabbrev]{cleveref}

%%%%%%%%%%%%%%%%%%%%%%%%%%%%%%%%
% THEOREMS
%%%%%%%%%%%%%%%%%%%%%%%%%%%%%%%%
\theoremstyle{plain}
\newtheorem{theorem}{Theorem}[section]
\newtheorem{proposition}[theorem]{Proposition}
\newtheorem{lemma}[theorem]{Lemma}
\newtheorem{corollary}[theorem]{Corollary}
\theoremstyle{definition}
\newtheorem{definition}[theorem]{Definition}
\newtheorem{assumption}[theorem]{Assumption}
\theoremstyle{remark}
\newtheorem{remark}[theorem]{Remark}

% Todonotes is useful during development; simply uncomment the next line
%    and comment out the line below the next line to turn off comments
%\usepackage[disable,textsize=tiny]{todonotes}
\usepackage[textsize=tiny]{todonotes}

\usepackage{rmc_cmds}

% The \icmltitle you define below is probably too long as a header.
% Therefore, a short form for the running title is supplied here:
\icmltitlerunning{Ricochet Monte Carlo}

\begin{document}

\twocolumn[
\icmltitle{Ricochet Monte Carlo: Dual Purpose \\ MCMC Sampling \& Nonlinear Programming}

\begin{icmlauthorlist}
    \icmlauthor{Abdi-Hakin Dirie}{n/a}
\end{icmlauthorlist}

\icmlaffiliation{n/a}{No affiliation}

\icmlcorrespondingauthor{Abdi-Hakin Dirie}{abdidirie0@gmail.com}

\icmlkeywords{Bayesian Inference, MCMC, optimization, nonlinear programming}

\vskip 0.3in
]

% this must go after the closing bracket ] following \twocolumn[ ...

% This command actually creates the footnote in the first column
% listing the affiliations and the copyright notice.
% The command takes one argument, which is text to display at the start of the footnote.
% The \icmlEqualContribution command is standard text for equal contribution.
% Remove it (just {}) if you do not need this facility.

\printAffiliationsAndNotice{Code for this paper is located at \url{https://github.com/aadah/RMC.jl}.}

\begin{abstract}
We introduce a novel dual-purpose algorithm for MCMC sampling and constrained
nonlinear optimization. The method simulates a particle bouncing across a
surface whose shape is determined by the target distribution or objective
function, using a notion of entropic dissipation that resembles simulated
annealing without any acceptance step. Similarly, the simple closed-form
parabolic trajectories means avoiding costly and potentially unstable numerical
integrations to compute trajectories. This enables a more efficient simulation,
reducing total evaluations of the target. Our method easily extends to handle
nonlinear constraints. We demonstrate the effectiveness of our method on several
distributions and nonlinear test functions.
\end{abstract}

\section{Introduction} \label{s:intro}
\subsection{Related work} \label{ss:related}
% - HMC
% - BPS
% - differences between both, such as
% with HMC:
% - HMC as an acceptance propability proportional to how well H was preserved, whereas RMC allows for annealing
%   - question: does annealing help with integration?
% - HMC treats distribution modes as separate gravitational well. RMC has one "planet"
% - HMC particle is frictionless puck on S, RMC particle bounces around S
% - no trajectory length L in RMC, in fact # of leapfrogs steps is hard to predict
% - between momentum refreshes, HMC gets at most 1 sample. RMC gets several
% with BPS:

\subsection{Outline of work} \label{ss:outline}

\section{Ricochet Monte Carlo} \label{s:rmc}

The intuition behind Ricochet Monte Carlo (RMC) is to treat our target
distribution or objective as a sort of hilly terrain across which a point
particle freely bounces along. The points where the particle collides with this
surface are candidate samples to either be accepted and rejected. The particle
will continue to bounce, until it settles in some ``valley'', at which time it is
picked up and tossed in a random direction to repeat the process.

Let \target{} be some probability distribution to sample from, defined as
\begin{equation*}
    \target{} \coloneqq \frac{\energy{}}{Z}
\end{equation*}
where $Z$ is some intractable normalizing constant and \energy{} is the
associated joint distribution, which is assumed tractable and known. We are
interested in generating samples $\thetai{} \sim \F$ that live in a continuous
$d$-dimensional vector space, where $i$ means the $i$th sample. We perform the
standard transformation as done in HMC and define our \emph{surface} as
\begin{equation*}
    \label{eq:surface}
    \surface{} \coloneqq - \log \energy{}.
\end{equation*}
If instead \target{} is a nonlinear objective function to minimize, then we
simply set $\surface{} \coloneqq \target{}$.

\subsection{Auxillary variables} \label{ss:auxillary}

For each value of \thetab{}, we associate an auxillary \emph{height} scalar
\h{}. We define a $(d+1)$-length \emph{position} vector $\q{} \coloneqq
[\thetab{}; \h{}]$. We also create an auxillary $(d+1)$-length \emph{position}
vector \p{}. We define functions
\begin{align*}
    \thetaof{\q{}} &\coloneqq  \thetab{} \\
    \hof{\q{}}     &\coloneqq  \h{}
\end{align*}
to respectively refer to the specific sample and height components of any
particular \q{}. Similarly, we overload \thetaof{\p{}} to be the vector of the
first $d$ entries of \p{}.

We maintain an invariant where the particle is always ``above'' the surface \S,
i.e.
\begin{equation}
    \label{eq:collision}
    \CSof{\q{}} \coloneqq \hof{\q{}} - \Sof{\thetaof{\q{}}} > 0.
\end{equation}
\cref{eq:collision} defines the \emph{surface collision} constraint \CS{}. This
terminology will be used to distinguish it from any later nonlinear constraints
that will be handled in the optimization formulation. All together, \q{}, \p{},
and \S{} fully describe the state of our system.

\subsection{Simulation with Hamiltonian dynamics} \label{ss:hamiltonian}

To simulate the particle's trajectory across the surface, we start with its
associated Hamiltonian equations. Given scalars $m > 0$ and $g > 0$, we define
\begin{align}
    H(\q{}, \p{}) &\coloneqq U(\q{}) + K(\p{}) \label{eq:H} \\
    U(\q{}) &\coloneqq m \cdot g \cdot \hof{\q{}} \label{eq:U} \\
    K(\p{}) &\coloneqq \frac{\lVert \p{} \rVert^2}{2m} \label{eq:K} .
\end{align}
We respectively call $m$ and $g$ the \emph{mass} and \emph{gravity}, and both
are hyperparameters of RMC. For convenience of notation, let $\g{} \coloneqq
[\mathbf{0}_d; g]$ where $\mathbf{0}_d$ is a $d$-length zero vector. The
dynamics of the particle are governed by the following system of differential
equations:
\begin{alignat}{4}
    &\frac{d\q{}}{dt} = &&\frac{\partial H}{\partial \p{}} = &&\frac{dK}{d\p{}} = &&\frac{\p{}}{m} \label{eq:dqdt} \\
    &\frac{d\p{}}{dt} = -&&\frac{\partial H}{\partial \q{}} = -&&\frac{dU}{d\q{}} = -&&m\g{}. \label{eq:dpdt}
\end{alignat}
In order to simulate the particle trajectory, HMC employs Euler's ``leapfrog''
integration method, which requires an acceptance step to handle cases when the
Hamiltonian $H$ diverges and thus isn't conserved. In RMC, because
\cref{eq:dqdt,eq:dpdt} don't involve what could be an arbitrarily complex
\surface{}, we can easily solve for the time-parametrized closed forms:
\begin{align}
    \qt{} &\coloneqq \qz{} + \frac{\pz{}}{m}t - \frac{\g{}}{2}t^2 \label{eq:qt} \\
    \pt{} &\coloneqq \pz{} - m\g{}t. \label{eq:pt}
\end{align}
We use \cref{eq:qt,eq:pt} to find the next candidate sample \thetai{} by
computing the earliest $t'$ when the particle collides with the \S{}. This
search is done using a two-step process that increases $t'$ in doubling
increments until a collision, and then does a binary search within a
time-bounded window to find the near-exact $t'$ representing the imminent point
of collision while still satisfying $\CSof{\qof{t'}} > 0$.
\cref{alg:temporalsearch} outlines the procedure in fuller detail.

\begin{algorithm}[tb]
   \caption{Find earliest $t'$ with imminent collision on \CS{}}
   \label{alg:temporalsearch}
\begin{algorithmic}
    \Procedure{TemporalSearch}{\CS, \qz{}, \pz{}, $g$, $m$, $\Delta_0$}
        \State $(\ts{}, \te{}) \gets (0, \Delta_0)$
    \Repeat \Comment{Double timestep in forward scan...}
        \State $\ts{} \gets \te{}$
        \State $\te{} \gets 2 \cdot \te{}$
    \Until{$\CSof{\qof{\te{}}} \leq 0$} \Comment{...until a collision.}

    \Repeat
        \State $\tm{} \gets \frac{1}{2} \left(\ts{} + \te{} \right)$
        \LComment {If temporal midpoint is a collision...}
        \If{$\CSof{\qof{\tm{}}} \leq 0$}
            \LComment{...move boundary endtime backward...}
            \State $\te{} \gets \tm{}$
        \Else
        \LComment{...else move boundary starttime forward.}
        \State $\ts{} \gets \tm{}$
        \EndIf
    \Until{$\te{} - \ts{} \leq \text{some small tolerance}$}

    \LComment{Since \ts{} still satisfies \CS{}, return it.}
    \State \Return \ts{} as $t'$
    \EndProcedure
\end{algorithmic}
\end{algorithm}

% TODO: Show how temporalsearch does O(log(n)) evaluations compared to HMC's O(n)

\subsection{Sampling at collisions} \label{ss:sampling}

Once $t'$ is had, we compute the position $\qu{t'} \coloneqq \qof{t'}$ and
momentum $\pu{t'} \coloneqq \pof{t'}$. The acceptance probability of the
candidate sample $\thetaof{\qu{t'}}$ is defined in terms of \emph{lateral
momentum}: ignoring the height component, if the angle of the reflection is
particularly sharp, we are more likely to reject it. The reasoning is that these
``hard'' bounces represent regions of low probability in \target{}. If we let
\begin{equation*}
    \mathbf{n} \coloneqq \frac{\nabla \CSof{\qu{t'}}}{\lVert \nabla \CSof{\qu{t'}} \rVert}
\end{equation*}
be the unit normal at \qu{t'}, we can define the momentum after reflecting off
the surface as
\begin{equation}
    \label{eq:p_after_simple}
    \overleftarrow{\p}_{t'} \coloneqq \pu{t'} - 2 \cdot \left<\pu{t'}, \mathbf{n}\right> \cdot \mathbf{n}.
\end{equation}
Our acceptance probability is thus
\begin{equation}
    \label{eq:accept}
    \Pr\left(\textsc{accept} \mid t'\right)
    \coloneqq \frac{1}{2}
    \left(1 + \frac{
        \left<\thetaof{\pu{t'}},\thetaof{\overleftarrow{\p}_{t'}}\right>
    }{
        \lVert \thetaof{\pu{t'}} \rVert \lVert \thetaof{\overleftarrow{\p}_{t'}} \rVert
    }\right).
\end{equation}
If accepted, then we set $\thetai \gets \thetaof{\qu{t'}}$. Whether the
sample is accepted or rejected, we still continue on with the simulation
by setting $(\qz{}, \pz{}) \gets (\qu{t'}, \pu{t'})$ in \cref{eq:qt,eq:pt}.

\subsection{Entropic dissipation} \label{ss:entropy}

Given a scalar $\epsilon \in (0,1]$, we modify \cref{eq:p_after_simple} slightly:
\begin{equation}
    \label{eq:p_after}
    \overleftarrow{\p}_{t'} \coloneqq \epsilon \cdot \left(\pu{t'} - 2 \cdot \left<\pu{t'}, \mathbf{n}\right> \cdot \mathbf{n}\right).
\end{equation}
The hyperparameter $\epsilon$ is known in the physics literature as the
\emph{coefficient of restitution}, a ratio of the velocities before and after an
inelastic collision. If $\epsilon = 1$, it is an elastic collision and we have
\cref{eq:p_after_simple} again. Repeated applications of \cref{eq:p_after} will
gradually remove energy from the system until what remains is accounted for in
the potential energy $U(\q{})$.

\begin{lemma}
    \label{lm:equilibrium}
    Let $t$ be the aggregate time elapsed across all bounces, with \qt{} and
    \pt{} being the position and momentum at this moment. As $t \rightarrow
    \infty$, \thetaof{\qt{}} will either be a local or global minimum of
    \surface{}.
\end{lemma}
\begin{proof}
    There is a point $\q{}^*$ such that $\Sof{\thetaof{\q{}^*}}$ is minimized.
    It must be the case then that $\CSof{\q{}^*} = 0$ (else we would shift the
    point downwards). Also, by \cref{eq:collision}, $\CSof{\qt{}} > 0$ always.
    Let $\Delta U(\qt{}) \coloneqq U(\qt{}) - U(\q{}^*)$ be a strictly positive
    quantity denoting the potential differential. We also observe per
    \cref{eq:K,eq:p_after} that $K(\overleftarrow{\p}(t)) = \epsilon^2
    K(\pt{})$. This implies
    \begin{equation*}
        \lim_{t \rightarrow \infty} K(\pt{}) = 0
    \end{equation*}
    and thus
    \begin{align*}
        \lim_{t \rightarrow \infty} H(\qt{}, \pt{}) &= U(\qt{}) \\
        &= U(\q{}^*) + \Delta U(\qt{}).
    \end{align*}
    If $K(\pt{}) = 0$ in the limit then the particle has ceased motion, and is
    resting on \S{} somewhere. If $\Delta U(\qt{}) = 0$, then $\qt{} = \q{}^*$
    and we have settled at the global minimum. Otherwise, we have found a local
    minimum.
\end{proof}

The result of \cref{lm:equilibrium} gives us a means of collecting values
\thetab{} that identify potential solutions of \target{} from which we can
select from at the end of the simulation. This is done by running the simulation
of bounces until the kinetic energy of the particle drops below some small
threshold $\eta > 0$. If that criterion is met, we collect a sample as part of a
separate collection of solutions distinct from our MCMC samples. These solutions
are therefore not subject to the acceptance step according to \cref{eq:accept}.

We must refresh momentum variable at the particle has ``puttered out''. We do
this by raising the particle to some random height above \S{} and sampling the
momentum randomly as well. For some \q{} and \p{} after a collision in which
$K(\p{}) < \eta$, we refresh according to the following:
\begin{align*}
    \Delta h &\sim \textsc{Rayleigh}(m) \\
    \q{} &\gets [\thetaof{\q}; \Sof{\thetaof{\q}} + \Delta h] \\
    \p{} &\sim \mathcal{N}(0, m\mathbf{I}_{d+1})
\end{align*}
where
\begin{equation*}
    \textsc{Rayleigh}(x \mid \sigma) \coloneqq \frac{x}{\sigma^2}\exp\left(\frac{-x^2}{2\sigma^2}\right)
\end{equation*}
is the Rayleigh distribution with scalar parameter $\sigma > 0$ and support $x
\geq 0$. This refresh scheme is visualized as raising the particle to some
random height above where it settled and tossing it in a random direction. The
raised height and strength of the toss is proportional to the mass $m$.

Between refreshes, we yield some undefined number of samples, different each
time and hard to predict.

\subsection{Handling boundary constraints} \label{ss:constraints}
% - motivation: some \theta_j could be a non-negative parameter in a graphical model, e.g. Beta
% - we could take the log of them, or treat them as an infinitely tall wall
% - constraint as 3-tuple (j, a, b) where a < b and j \in {1,...,d}
% - normal simply zero vector with 1 at location j
% - reflection is just multiple p_j by -1
% - we *don't* sample if collision is on boundary wall, and not S

\subsection{Initialization} \label{ss:initialization}

\subsection{Full algorithm} \label{ss:algo}
% - what are the hyperparameters
% - initialization
% - starting height?
% - clamp q to any boundary constraints
% - actual pseudocode

\section{Experiments}
\subsection{Distribution sampling}
\subsubsection{Single gaussian}
% - 1D, 2D, and 1000D cases
% - visualize estimated samples and trajectories

\subsubsection{Mixture of varied gaussians}
\subsubsection{Lopsided distributions}
\subsubsection{Graphical model with boundary constraints}
% - use coin flip model with Beta prior on flip prob \phi. \theta = [a, b, \theta]
% - sub-exp 1: a, b fully continuous and we use exp(-a) and exp(-b) as Beta params
% - sub-exp 2: a, b be as in Beta prior but constrained, need to do ricocheting

\subsection{Nonlinear programming}

\section{Theoretical analysis \& discussion}
% - runtime and space
% - ergodicity / time-reversibility / detailed balance
% - relation between hyperparameters and convergence guarantees
% - hyperparameter tuning

\subsection{Potential extensions}
% - stochastic gradient, i.e. S changes with each batch X, faster compute of S
% - step-wise distributions where walls are finitely tall
% - n-nary boundary constraints (e.g. uniform dist a<b) and how to reflect/resolve
% - more efficient collision checking (very expensive if S(\theta) is non-trivial)

\section{Conclusion}

% - function name
% - function dimension
% - type (constrained or unconstrained)
% - pct. solutions "close enough"
% - median function evaluations
% - # evaluations for best solution
% - pct. runs ended in "max evals allowed"
% - pct. runs ended in misc errors

\begin{table*}
    \centering
    \begin{tabular*}{\textwidth}{@{\extracolsep{\fill}}lcccc@{}}
    \toprule
    Column 1 & Column 2 & Column 3 & Column 4 & Column 5 \\
    \midrule
    Row 1 & Data & Data & Data & Data \\
    Row 2 & Data & Data & Data & Data \\
    Row 3 & Data & Data & Data & Data \\
    \bottomrule
    \end{tabular*}
    \caption{Your table caption here}
    \label{tab:your_label_here}
    \end{table*}


%-------------------------------------------------------------------------------

% In the unusual situation where you want a paper to appear in the
% references without citing it in the main text, use \nocite
\nocite{langley00}

\bibliography{rmc_icml}
\bibliographystyle{icml2023}


%%%%%%%%%%%%%%%%%%%%%%%%%%%%%%%%%%%%%%%%%%%%%%%%%%%%%%%%%%%%%%%%%%%%%%%%%%%%%%%
%%%%%%%%%%%%%%%%%%%%%%%%%%%%%%%%%%%%%%%%%%%%%%%%%%%%%%%%%%%%%%%%%%%%%%%%%%%%%%%
% APPENDIX
%%%%%%%%%%%%%%%%%%%%%%%%%%%%%%%%%%%%%%%%%%%%%%%%%%%%%%%%%%%%%%%%%%%%%%%%%%%%%%%
%%%%%%%%%%%%%%%%%%%%%%%%%%%%%%%%%%%%%%%%%%%%%%%%%%%%%%%%%%%%%%%%%%%%%%%%%%%%%%%
\newpage
\appendix
\onecolumn
\section{You \emph{can} have an appendix here.}

You can have as much text here as you want. The main body must be at most $8$ pages long.
For the final version, one more page can be added.
If you want, you can use an appendix like this one, even using the one-column format.
%%%%%%%%%%%%%%%%%%%%%%%%%%%%%%%%%%%%%%%%%%%%%%%%%%%%%%%%%%%%%%%%%%%%%%%%%%%%%%%
%%%%%%%%%%%%%%%%%%%%%%%%%%%%%%%%%%%%%%%%%%%%%%%%%%%%%%%%%%%%%%%%%%%%%%%%%%%%%%%


\end{document}


% This document was modified from the file originally made available by
% Pat Langley and Andrea Danyluk for ICML-2K. This version was created
% by Iain Murray in 2018, and modified by Alexandre Bouchard in
% 2019 and 2021 and by Csaba Szepesvari, Gang Niu and Sivan Sabato in 2022.
% Modified again in 2023 by Sivan Sabato and Jonathan Scarlett.
% Previous contributors include Dan Roy, Lise Getoor and Tobias
% Scheffer, which was slightly modified from the 2010 version by
% Thorsten Joachims & Johannes Fuernkranz, slightly modified from the
% 2009 version by Kiri Wagstaff and Sam Roweis's 2008 version, which is
% slightly modified from Prasad Tadepalli's 2007 version which is a
% lightly changed version of the previous year's version by Andrew
% Moore, which was in turn edited from those of Kristian Kersting and
% Codrina Lauth. Alex Smola contributed to the algorithmic style files.
