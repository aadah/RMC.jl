\section{Ricochet Monte Carlo} \label{s:rmc}

The intuition behind Ricochet Monte Carlo (RMC) is to treat our target
distribution or objective as a sort of hilly terrain across which a point
particle freely bounces along. The points where the particle collides with this
surface are candidate samples to either be accepted and rejected. The particle
will continue to bounce, until it settles in some ``valley'', at which time it is
picked up and tossed in a random direction to repeat the process.

Let \target{} be some probability distribution to sample from, defined as
\begin{equation*}
    \target{} \coloneqq \frac{\energy{}}{Z}
\end{equation*}
where $Z$ is some intractable normalizing constant and \energy{} is the
associated joint distribution, which is assumed tractable and known. We are
interested in generating samples $\thetai{} \sim \F$ that live in a continuous
$d$-dimensional vector space, where $i$ means the $i$th sample. We perform the
standard transformation as done in HMC and define our \emph{surface} as
\begin{equation*}
    \label{eq:surface}
    \surface{} \coloneqq - \log \energy{}.
\end{equation*}
If instead \target{} is a nonlinear objective function to minimize, then we
simply set $\surface{} \coloneqq \target{}$.

\subsection{Auxillary variables} \label{ss:auxillary}

For each value of \thetab{}, we associate an auxillary \emph{height} scalar
\h{}. We define a $(d+1)$-length \emph{position} vector $\q{} \coloneqq
[\thetab{}; \h{}]$. We also create an auxillary $(d+1)$-length \emph{position}
vector \p{}. We define functions
\begin{align*}
    \thetaof{\q{}} &\coloneqq  \thetab{} \\
    \hof{\q{}}     &\coloneqq  \h{}
\end{align*}
to respectively refer to the specific sample and height components of any
particular \q{}. Similarly, we overload \thetaof{\p{}} to be the vector of the
first $d$ entries of \p{}.

We maintain an invariant where the particle is always ``above'' the surface \S,
i.e.
\begin{equation}
    \label{eq:collision}
    \CSof{\q{}} \coloneqq \hof{\q{}} - \Sof{\thetaof{\q{}}} > 0.
\end{equation}
\cref{eq:collision} defines the \emph{surface collision} constraint \CS{}. This
terminology will be used to distinguish it from any later nonlinear constraints
that will be handled in the optimization formulation. All together, \q{}, \p{},
and \S{} fully describe the state of our system.

\subsection{Simulation with Hamiltonian dynamics} \label{ss:hamiltonian}

To simulate the particle's trajectory across the surface, we start with its
associated Hamiltonian equations. Given scalars $m > 0$ and $g > 0$, we define
\begin{align}
    H(\q{}, \p{}) &\coloneqq U(\q{}) + K(\p{}) \label{eq:H} \\
    U(\q{}) &\coloneqq m \cdot g \cdot \hof{\q{}} \label{eq:U} \\
    K(\p{}) &\coloneqq \frac{\lVert \p{} \rVert^2}{2m} \label{eq:K} .
\end{align}
We respectively call $m$ and $g$ the \emph{mass} and \emph{gravity}, and both
are hyperparameters of RMC. For convenience of notation, let $\g{} \coloneqq
[\mathbf{0}_d; g]$ where $\mathbf{0}_d$ is a $d$-length zero vector. The
dynamics of the particle are governed by the following system of differential
equations:
\begin{alignat}{4}
    &\frac{d\q{}}{dt} = &&\frac{\partial H}{\partial \p{}} = &&\frac{dK}{d\p{}} = &&\frac{\p{}}{m} \label{eq:dqdt} \\
    &\frac{d\p{}}{dt} = -&&\frac{\partial H}{\partial \q{}} = -&&\frac{dU}{d\q{}} = -&&m\g{}. \label{eq:dpdt}
\end{alignat}
In order to simulate the particle trajectory, HMC employs Euler's ``leapfrog''
integration method, which requires an acceptance step to handle cases when the
Hamiltonian $H$ diverges and thus isn't conserved. In RMC, because
\cref{eq:dqdt,eq:dpdt} don't involve what could be an arbitrarily complex
\surface{}, we can easily solve for the time-parametrized closed forms:
\begin{align}
    \qt{} &\coloneqq \qz{} + \frac{\pz{}}{m}t - \frac{\g{}}{2}t^2 \label{eq:qt} \\
    \pt{} &\coloneqq \pz{} - m\g{}t. \label{eq:pt}
\end{align}
We use \cref{eq:qt,eq:pt} to find the next candidate sample \thetai{} by
computing the earliest $t'$ when the particle collides with the \S{}. This
search is done using a two-step process that increases $t'$ in doubling
increments until a collision, and then does a binary search within a
time-bounded window to find the near-exact $t'$ representing the imminent point
of collision while still satisfying $\CSof{\qof{t'}} > 0$.
\cref{alg:temporalsearch} outlines the procedure in fuller detail.

\begin{algorithm}[tb]
   \caption{Find earliest $t'$ with imminent collision on \CS{}}
   \label{alg:temporalsearch}
\begin{algorithmic}
    \Procedure{TemporalSearch}{\CS, \qz{}, \pz{}, $g$, $m$, $\Delta$}
        \State $(\ts{}, \te{}) \gets (0, \Delta)$
    \Repeat \Comment{Double timestep in forward scan...}
        \State $\ts{} \gets \te{}$
        \State $\te{} \gets 2 \cdot \te{}$
    \Until{$\CSof{\qof{\te{}}} \leq 0$} \Comment{...until a collision.}

    \Repeat
        \State $\tm{} = \frac{1}{2} \left(\ts{} + \te{} \right)$
        \LComment {If temporal midpoint is a collision...}
        \If{$\CSof{\qof{\tm{}}} \leq 0$}
            \LComment{...move boundary endtime backward...}
            \State $\te{} \gets \tm{}$
        \Else
        \LComment{...else move boundary starttime forward.}
        \State $\ts{} \gets \tm{}$
        \EndIf
    \Until{$\te{} - \ts{} \leq \text{some small tolerance}$}

    \LComment{Since \ts{} still satisfies \CS{}, return it.}
    \State \Return \ts{} as $t'$
    \EndProcedure
\end{algorithmic}
\end{algorithm}

% TODO: Show how temporalsearch does O(log(n)) evaluations compared to HMC's O(n)

\subsection{Sampling at collisions} \label{ss:sampling}

Once $t'$ is had, we compute the position $\qu{t'} \coloneqq \qof{t'}$ and
momentum $\pu{t'} \coloneqq \pof{t'}$. The acceptance probability of the
candidate sample $\thetaof{\qu{t'}}$ is defined in terms of \emph{lateral
momentum}: ignoring the height component, if the angle of the reflection is
particularly sharp, we are more likely to reject it. The reasoning is that these
``hard'' bounces represent regions of low probability in \target{}. If we let
\begin{equation*}
    \mathbf{n} \coloneqq \frac{\nabla \CSof{\qu{t'}}}{\lVert \nabla \CSof{\qu{t'}} \rVert}
\end{equation*}
be the unit normal at \qu{t'}, we can define the momentum after reflecting off
the surface as
\begin{equation}
    \label{eq:p_after_simple}
    \overleftarrow{\p}_{t'} \coloneqq \pu{t'} - 2 \cdot \left<\pu{t'}, \mathbf{n}\right> \cdot \mathbf{n}.
\end{equation}
Our acceptance probability is thus
\begin{equation}
    \label{eq:accept}
    \Pr\left(\textsc{accept} \mid t'\right)
    \coloneqq \frac{1}{2}
    \left(1 + \frac{
        \left<\thetaof{\pu{t'}},\thetaof{\overleftarrow{\p}_{t'}}\right>
    }{
        \lVert \thetaof{\pu{t'}} \rVert \lVert \thetaof{\overleftarrow{\p}_{t'}} \rVert
    }\right).
\end{equation}
If accepted, then we set $\thetai \gets \thetaof{\qu{t'}}$. Whether the
sample is accepted or rejected, we still continue on with the simulation
by setting $(\qz{}, \pz{}) \gets (\qu{t'}, \pu{t'})$ in \cref{eq:qt,eq:pt}.

\subsection{Entropic dissipation} \label{ss:entropy}

Given a scalar $\epsilon \in (0,1]$, we modify \cref{eq:p_after_simple} slightly:
\begin{equation}
    \label{eq:p_after}
    \overleftarrow{\p}_{t'} \coloneqq \epsilon \cdot \left(\pu{t'} - 2 \cdot \left<\pu{t'}, \mathbf{n}\right> \cdot \mathbf{n}\right).
\end{equation}
The value $\epsilon$ is known in the physics literature as the \emph{coefficient
of restitution}, a ratio of the velocities before and after an inelastic
collision. If $\epsilon = 1$, it is an elastic collision and we have
\cref{eq:p_after_simple} again. Repeated applications of \cref{eq:p_after} will
gradually remove energy from the system until what remains is accounted for in
the potential energy $U(\q{})$.

\begin{lemma}
    \label{lm:equilibrium}
    Let $t$ be the aggregate time elapsed across all bounces, with \qt{} and
    \pt{} being the position and momentum at this moment. As $t \rightarrow
    \infty$, \thetaof{\qt{}} will either be a local or global minimum of
    \surface{}.
\end{lemma}
\begin{proof}
    There is a point $\q{}^*$ such that $\Sof{\thetaof{\q{}^*}}$ is minimized.
    It must be the case then that $\CSof{\q{}^*} = 0$ (else we would shift the
    point downwards). Also, by \cref{eq:collision}, $\CSof{\qt{}} > 0$ always.
    Let $\Delta U(\qt{}) \coloneqq U(\qt{}) - U(\q{}^*)$ be a strictly positive
    quantity denoting the potential differential. We also observe per
    \cref{eq:K,eq:p_after} that $K(\overleftarrow{\p}(t)) = \epsilon^2
    K(\pt{})$. This implies
    \begin{equation*}
        \lim_{t \rightarrow \infty} K(\pt{}) = 0
    \end{equation*}
    and thus
    \begin{align*}
        \lim_{t \rightarrow \infty} H(\qt{}, \pt{}) &= U(\qt{}) \\
        &= U(\q{}^*) + \Delta U(\qt{}).
    \end{align*}
    If $K(\pt{}) = 0$ in the limit then the particle has ceased motion, and is
    resting on \S{} somewhere. If $\Delta U(\qt{}) = 0$, then $\qt{} = \q{}^*$
    and we have settled at the global minimum. Otherwise, we have found a local
    minimum.
\end{proof}

The result of \cref{lm:equilibrium} gives us a means of collecting values
\thetab{} that identify potential solutions of \target{} from which we can
select from at the end of the simulation. This is done by running the simulation
of bounces until the kinetic energy of the particle drops below some small
threshold $\eta > 0$. If that criterion is met, we collect a sample as part of a
separate collection of solutions distinct from our MCMC samples. These solutions
are therefore not subject to the acceptance step according to \cref{eq:accept}.

We must refresh momentum variable at the particle has ``puttered out''. We do
this by raising the particle to some random height above \S{} and sampling the
momentum randomly as well. For some \q{} and \p{} after a collision in which
$K(\p{}) < \eta$, we refresh according to the following:
\begin{align*}
    \Delta h &\sim \textsc{Rayleigh}(m) \\
    \q{} &\gets [\thetaof{\q}; \Sof{\thetaof{\q}} + \Delta h] \\
    \p{} &\sim \mathcal{N}(0, m\mathbf{I}_{d+1}).
\end{align*}

Between refreshes, we yield some undefined number of samples, different each
time and hard to predict.

\subsection{Handling boundary constraints} \label{ss:constraints}
% - motivation: some \theta_j could be a non-negative parameter in a graphical model, e.g. Beta
% - we could take the log of them, or treat them as an infinitely tall wall
% - constraint as 3-tuple (j, a, b) where a < b and j \in {1,...,d}
% - normal simply zero vector with 1 at location j
% - reflection is just multiple p_j by -1
% - we *don't* sample if collision is on boundary wall, and not S

\subsection{Initialization} \label{ss:initialization}

\subsection{Full algorithm} \label{ss:algo}
% - what are the hyperparameters
% - initialization
% - starting height?
% - clamp q to any boundary constraints
% - actual pseudocode
